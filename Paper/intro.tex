%%%%%%%%%%%%%%%%%%%%%%%%%%%%%%%%%%%%%%%%%%%%%%%%%%%%%%%%%%%%%%%%%
\section{Introduction}
The structure of the brain influences availability and access to the memories stored therein. Communication between neurons allows animals to think, move, and store or recall memories. It seems plausible that the structure of an artificial neural network would similarly affect the ability of that neural network to store and process information. Building off this idea, we suggest a new model of neural network influenced by Perin, Berger, and Markram (2011) who proposed that the brain is composed of a hierarchy of connected components with specific associated computational functions. 

We implemented this model within the framework of deep learning inspired by Hinton (2007).  More specifically, we describe a model of deep learning based on the use of acyclic clusters which mimic multiple layers of neurons. Additionally, these clusters themselves are layered. We found that this model did not significantly increase the ability of our network to generate or classify handwritten digits, nor did it increase the computational efficiency of those tasks.
%\label{sec:Introduction}
%%%%%%%%%%%%%%%%%%%%%%%%%%%%%%%%%%%%%%%%%%%%%%%%%%%%%%%%%%%%%%%%%

Some text, cite this~\cite{ASBMI,Acharya07}, and also Figure~\ref{fig:something}.