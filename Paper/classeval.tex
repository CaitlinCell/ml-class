\section{Classification Evaluation} % (fold)
\label{sec:classification_evaluation}

The IPC classification system was initially created in 1968. Although the subcategories have been changed an expanded since that time, the 8 main top-level categories have not been altered. In the 45 years since the categories were initially established the technological landscape has changed greatly and they may not be as relevant to current and future patent applications. We used several unsupervised techniques and metrics to attempt to evaluate the suitability of the 8 top-level IPC categories.

\ref{tab:skew}Table XXXX shows the proportion of each label class found in \emph{jagged-40000}. We can see that the distribution of patents between the categories is highly skewed, with <ADD CONTENT>. As a first approximation it is desirable that each category occur with roughly equal frequency. Extreme imbalances in the proportions of categories in a sample implies that some are too specific and others are too general. This initial intuition is borne out by metrics of topic coherence that have been used to evaluate the quality of topics generated by unsupervised techniques. We used a coherence metric that compares document frequency with co-document frequency\cite{Mimno_optimizingsemantic}. Specifically, if we define $D(v)$ to be the number of documents in a corpus that contain a given word $v$, $D(v, v')$ to be the number of documents in a corpus that contain both $v$ and $v'$, and $V^{(t)}_M$ to be the $M$ most common words in topic $t$, then we define the coherence of a topic $t$ to be:
\begin{align*}
	C(t;V^{(t)}) = \sum_{m=2}^M \sum_{l=1}^{m-1} \log \frac{D(v^{(t)}_m, v^{(t)}_l)}{D(v^{(t)}_l)}
\end{align*}



\begin{tablehere}
	\label{tab:skew}
	\centering
	\caption{Proportion of each class in \emph{jagged-40000}}
\end{tablehere}

Table of skew


Table of topics


Coherence

% section classification_evaluation (end)