\section{Results}
\subsection{Experiment 1: Baseline}
In the first experiment, our dataset consisted of approximately 3500 patents.  The data was evenly split over the each of the 7 categories.
The training data consisted of 70\% of the data while the testing data consisted of 30\% of the data.  The following is the result of the testing
set.
\\
\\Support Vector Machine: 49.3\% Accuracy
\\Maximum Entropy: 46\% Accuracy
\\
\\Because the data is evenly split, a random choice would expect the results to be near 14\%, which means that even with these
base algorithms, the results are above chance.
\subsection{Experiment 2: Dimensionality Reduction}
In the second experiment, we selected two different dimensionality reduction techniques, principle component analysis, and laplacian eigenmap.
The two dimensionality techniques were then applied to the dataset from Experiment 1.  The results are shown in two tables below.
\\
\begin{table}[h]
\caption{Maximum Entropy with Dimensioanlity Reduction}
\centering
\begin{tabular}{c c c}
Features & Laplacian Eigenmap & PCA Accuracy \\
3 & 15 & 18 \\
10 & 24 & 28 \\
50 & 30 & 52 \\
100 & 35 & 52 \\
1000 & 44 & 59 \\
\end{tabular}
\end{table}
\\
\begin{table}[h]
\caption{SVM with Dimensioanlity REduction}
\centering
\begin{tabular}{c c c}
Features & Laplacian Eigenmap & PCA Accuracy \\
3 & 17 & 18 \\
10 & 18 & 26 \\
50 & 26 & 26 \\
100 & 31 & 32 \\
1000 & 35 & 42 \\
\end{tabular}
\end{table}
\\
\\ Visualization of these tables are shown here.
\\ Insert graphs here
\\
The results are unclear.  This is due to the fact that compared with the baseline Maximum Entropy had an icnrease in its testing accuracy, however
the support vector machine had a decrease in its testing accuracy.  Therefore, the results of dimensionality reduction appears to be dependent
on the classifier which was used.
\\
\\ Graphs of these results are shown below.
Insert Graphs here!
\\
\subsection{Experiment 3}
In the third experiment we attempted to increase the sample size of our dataset to determine if this would noticeably increase accuracy when
compared against the baseline.  The dataset was increased to 20000 and 40000.  The results are shown below.
\\
\\
\\
\\
\\
\\
\\
\\
\begin{table}[h]
\caption{Dataset Incrase}
\centering
\begin{tabular}{c c c}
Dataset & Classifier & Testing Accuracy \\
3500 & SVM & 49 \\
20000 & SVM & 70 \\
40000 & SVM & Timed Out \\
3500 & MaxEnt & 46 \\
20000 & MaxEnt & ?? \\
40000 & MaxEnt & 70 \\
\end{tabular}
\end{table}
\\
These results imply that increasing the dataset did greatly improve the accuracy on the testing set.

\subsection{Experiment 4}
We then changed the features of dataset to....
\\SVM 40000 new features: 72\%
\\Maximum Entropy new features: 73\% 


\subsection{Experiment 5}
Are we still doing an unsupervised portion?